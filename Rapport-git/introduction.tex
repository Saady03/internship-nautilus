De nos jours,la technologie des conteneurs reste la plus déployée pour la mise en place des outils dédiés au Devops. les Dockers rendent  la tache facile aux développeurs car ils permettent aux applications d'etre indépendantes des infrastructures.Mais avant d'aller dans le sujet 
Parlons d’abord virtualisation.
La virtualisation est un processus qui permet de créer d'autres machines virtuelles ayant les caractéristiques d'une machine physique et qui emploient les ressources informatiques de cette dernière(mémoire RAM,CPU etc...).Il s’agit de créer un système hôte pour qu’il ait les mêmes propriétés que ce dernier
%\vspace{0.5cm}
Une \textbf{machine virtuelle} : c’est une solution qui vient rajouter de la puissance aux pc ,aux hôtes pour effectuer les taches lorsque celles ci demandent assez de ressources.Elle permet donc d'exécuter plusieurs applications sur la même machine par exemple.
%\vspace{0.5cm}
Un \textbf{hyperviseur}:c'est une couche logicielle qui permet de gérer l'allocation des ressources.
Mais il arrive que souvent les applications qui sont installées  ne consomment pas toutes les ressources disponibles sur les machines virtuelles.Ce qui fait donc du gaspillage.Cependant, les hyperviseurs de machines virtuelles reposent sur une émulation du hardware, et requièrent donc beaucoup de puissance de calcul. Pour remédier à ce problème, de nombreuses firmes se tournent vers les containers, et par extension vers Docker.Avec Docker , les applications peuvent s'exécuter au sein du système d'exploitation mais de manière virtuelle et isolée,ce qui n'est pas le cas des machines virtuelles.
%Alors que la virtualisation consiste à exécuter de nombreux systèmes d’exploitation sur un seul et même système, les containers se partagent le même noyau de système d’exploitation et isolent les processus de l’application du reste du système.
%  les hyperviseurs doivent créer une copie complète du système d’exploitation qui fonctionne sur un espace matériel virtuel.sur un espace matériel virtuel. L’hyperviseur est donc responsable de tous les échanges de données. Exécuter plusieurs machines virtuelles sur un même serveur demande de grosses performances et un nombre suffisant de ressources pour assumer plusieurs machines virtuelles.