\vspace{0.5cm}
Git  est un logiciel décentralisé qui assure le contrôle de versions des fichiers.Il est un outil utilisé pour :
\begin{itemize}
	\item Répondre à des problèmes de perte de projets
	\item Offrir la contribution de projets
	\item Gérer de manière fluide le versionning des fichiers,gérer l'évolution du contenu d'une arborescence.
\end{itemize}
Il stocke au fur et à mesure qu'un fichier est modifié .
\section*{Les commandes de base}
\begin{itemize}
	\item \textbf{git init}:cette commande permet de créer un répertoire vide ou réexecute un existant afin de charger de nouveaux fichiers.
	Au cours d'un travail, les fichiers peuvent varier d'un état à un autre.Ainsi nous avons les états untracked,tracked,unmodified,modified et staged.Lorsque le fichier est crée et n'est pas ajouté il est non suivi,l'ajout d'un \textbf{ git add} le rend suivi soit \textbf{tracked}  puis indexé.Quant aux fichiers ajoutés,ils passent en phase d'indexation(\textbf{staged}).Une fois modifiés ils peuvent être rajoutés afin d'être indexés ou soit garder l'état modifié.Si ils sont ajoutés et validés ils sont donc non modifiés.Certaines commandes de base sont utilisées afin de manipuler les fichiers.Bien avant d'en parler nous allons parler de commit,validation de fichiers etc..  
	\item \textbf{Commit}: Il n'est rien d'autre que le changement effectué sur un fichier au cours du développement.Il contient un pointeur qui pointe sur le contenu du fichier indexé,le nom de l'auteur,son email,la description du commit puis des pointeurs des commits  parents pour former un graphe d'historique.La commande utilisée est : git commit -m "description".  
\end{itemize}
\section*{Le Repo distant} 
Git est aussi utilisé pour envoyer son travail sur un serveur dédié.Dans ce cas on parle de repertoire distant,ce dernier peut etre dupliqué ou cloné par d'autres developpeurs afin de permettre la contribution.On utilise la commande git clone pour cloner le dépôt distant:\textbf{git clone /chemin ou git clone file:///chemin}
via un réseau local.
\begin{itemize}
	\item
	\item 
	\item  
\end{itemize}


